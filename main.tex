\documentclass{beamer}
\usepackage[utf8]{inputenc}

\usepackage{amsthm, amsmath, mathtools, amssymb, faktor, geometry, units, mathrsfs, tikz-cd, Macros}

\usetikzlibrary{arrows}

\title{O-minimal structures}

\begin{document}

\begin{frame}
    \titlepage
\end{frame}

\section{Introduction to model theory}

\begin{frame}{Models}
    \begin{definition}
        A Model  $\mathcal{M}$ consists of a collection of objects $M$, some relations and functions on those objects, along with some constants.
    \end{definition}
    
\begin{example}
    $(\mathbb{N}, <, s, 0)$
     \begin{itemize}
         \item Our domain is just the natural numbers $\mathbb{N} = \{0, 1, 2, \ldots \}$
         \item $<$ is a relation, where $1 < 2$, $2 < 3$ and so on
         \item $s:\mathbb{N} \to \mathbb{N}$ is a function adding 1 to each number, so $s(0) = 1$, $s(1) = 2$
         \item We have $0$ as a constant (Don't worry too much about this.)
     \end{itemize}
     Another example is $(\mathbb{Q}, <)$, where the domain is the rational numbers, and $<$ means what you expect.
\end{example}
\end{frame}

\begin{frame}{Formulas}
    Think of formulas as polynomials with variables $x,y,z \ldots$ and constants from our structure. But instead of $x^n$ and $x + y$ we are restricted to the operations from the model. So for example, in $\mathcal{N} = (\mathbb{N}, <, s, 0)$, formulas would look like this:
    \begin{itemize}
        \item $s(x) = 1$
        \item $x > s(z)$
        \item $s(s(x)) < 0$
    \end{itemize}
    We can also have logical operators, like $\land, \lor, \neg$, which correspond to and, or, not. So we could have:
    \begin{itemize}
        \item $\neg(s(x) = 1)$
        \item $(s(5) = s(z)) \land (s(x) < 0)$
    \end{itemize}
\end{frame}

\begin{frame}{More Formulas}
    We can also use $\forall x$ and $\exists x$, which correspond with 'for every $x$' and 'there exists some $x$'. This gives even more formulas:
    
    \begin{itemize}
        \item $\forall x(s(x) = 0)$
        \item $\exists x \forall y (s(x) = y)$
    \end{itemize}
    
    Where it makes sense, a formula can be true or false in a structure. For example, the following are true in $\mathcal{N} = (\mathbb{N}, <, s, 0)$:
    \begin{itemize}
        \item $1 < 2$
        \item $\forall x (x > 0 \lor x = 0)$
        \item $\forall x (x = 0 \lor \exists y(s(y) = x))$
    \end{itemize}
    
    The following are false:
    \begin{itemize}
        \item $2 < 1$
        \item $(2 < 1) \land (1 < 2)$
        \item $\forall x \exists y( s(y) = x)$
    \end{itemize}
    
    If a formula $\phi$ is true in a model $\mathcal{M}$, then we write:
    
    $$ \mathcal{M} \models \phi$$
\end{frame}

\section{Definability}

\begin{frame}
    \begin{definition}
        Let $\mathcal{M} = (M, \ldots)$ be a structure, $\phi(\overline{x}, \overline{a})$ be a formula with free variables $\overline{x}$ and constants $\overline{a} \in M^n$. The set defined by $\phi(\overline{x}, \overline{a})$ is:
        
        $$ \phi(\mathcal{M}^k, \overline{a}) \coloneqq \{x \in M \mid \mathcal{M} \models \phi(\overline{x}, \overline{a}) \}$$
        
        We say $X \subseteq M^k$ is \textbf{definable} is there's some formula $\phi(\overline{x}, \overline{a})$ such that $X = \phi(\mathcal{M}^k, \overline{a})$
    \end{definition}
    
    \begin{example}
         Let $\mathcal{N} = (\mathbb{N}, <, +)$, then:
         
         \begin{itemize}
             \item $x = x$ defines the entirety of $\mathbb{N}$, including $0$!!!
             \item $\phi(x) \coloneqq \exists y (y < x)$ defines everything but $0$.
             \item $\psi(x) \coloneqq \exists y (y + y = x)$ defines all the even numbers.
             \item $\chi(x) \coloneqq \forall y (y + y \not = x)$ defines all the odd numbers.
             \item What does $\gamma (x) \coloneqq \exists y(y + y + y =x)$ define?
         \end{itemize}
    \end{example}
\end{frame}
    
\begin{frame}{Definable sets in $(\mathbb{Q}, <)$}
    Let $\mathcal{M} = (\mathbb{Q}, <)$, what do all the \textbf{quantifier free} definable sets look like? (So sets defined by formulas with no $\forall, \exists$).
    
    \begin{itemize}
        \item We have all points $q \in \mathbb{Q}$, given by $x = q$
        \item We have all intervals $(a,b)$, given by $a < x \land x < b$
        \item We also have half lines $(a, +\infty), (-\infty, b)$, given by $x > a$ and $x < b$.
    \end{itemize}
    
    Since the only operation on our structure was $<$, this exhausts all possible quantifier free definable sets! Specifically, any quantifier free definable set is going to be a finite union of the sets above. But we still haven't considered sets defined using quantifiers. Luckily, we don't have to:
    
    \begin{theorem}
        Any formula with quantifiers is equivalent to one without any in the structure $(\mathbb{Q}, <)$.
    \end{theorem}
\end{frame}

\section{O-minimality}

\begin{frame}
    \begin{definition}
        Let $\mathcal{M} = (M, < ,\ldots)$ be a structure such that $(M, <)$ is a \textbf{dense linear order without endpoints}. We say $\mathcal{M}$ is o-minimal if for any formula $\phi(x, \overline{a})$, where $\overline{a}$ is a tuple of constants in $M$, the set:
        
        $$ \{x \in M \mid \mathcal{M} \models \phi(x, \overline{a}) \}$$ is a \textbf{finite} union of points and intervals.
    \end{definition}

   \begin{example}
        \begin{itemize}
            \item $(\mathbb{Q}, <)$, as shown earlier
            \item $\mathbb{R} = (\mathbb{R}, +, - ,\cdot, 0, 1, <)$, since this structure also has quantifier elimination so we only need to consider the sets defined by $f(x) > 0$ and $f(x) = 0$, which can be shown to be finite collections of intervals.
            \item $(\mathbb{Q}, +, - ,\cdot, 0, 1, <)$ is \textbf{not} o-minimal. since the set defined by $x^2 < 2$ is not an interval in in $\mathbb{Q}$.
        \end{itemize}
    \end{example}
\end{frame}

\begin{frame}{Definable sets in more than one dimension}
    We know that o-minimality must place some restrictions on sets defined in $M^k$ for $k > 1$, since if $X \subseteq M^k$ is definable then so is $\pi(X) \subseteq M$, where $\pi: M^k \to M$ is a projection.
    
    For example, the set $ X = \{(0, z) \in \mathbb{R}^2 \mid z \in \mathbb{Z} \}$ can't be definable in $\mathcal{R}$, else $\pi(X) = \mathbb{Z}$ is definable too breaking o-minimality. Can we make this idea more precise?
    
    \begin{theorem}
        Let $\mathcal{M} = (M, <, +, 0, \ldots)$ be o-minimal with a group structure. Then for any definable function $f: M \to M$, there is a partition of $M$ into finitely many intervals $\{P_1, \ldots, P_n\}$ such that $f:P_i \to M$ is monotone and continuous.
    \end{theorem}
\end{frame}

\begin{frame}
    \begin{example}
         \begin{itemize}
             \item $f(x) = sin(x)$ is not definable since it's always going from increasing to decreasing. An appropriate partition would need to contain the intervals $(-\pi, \pi), (\pi, 2\pi), (2\pi, 3\pi) \ldots$. 
         \end{itemize}
    \end{example}
\end{frame}

\begin{frame}{Cell decomposition}
    \begin{itemize}
        \item Any interval $(a,b)$ is a cell ($a, b$ could possibly be $\pm \infty$).
        \item If $X \subseteq M^k$ is a cell, and $f,g:X \to M$ are definable function, then both:
            \begin{itemize}
                \item $\Gamma(f) \coloneqq \{(x, f(x)) \in M^{k+1} \mid x \in X \}$
                \item $(f,g)_X \coloneqq \{(x, y) \in M^{k+1} \mid f(x) < y < g(x) \}$
            \end{itemize}
        are cells
    \end{itemize}
    
    These are the building blocks for all definable sets in o-minimal structures:
    
    \begin{theorem}
        Let $\mathcal{M} = (M, <, +, 0, \ldots)$ be o-minimal with a group structure, and $X \subseteq M^k$ be definable. Then there is a partition of $X$ into finitely many cells.
    \end{theorem}
\end{frame}

\end{document}
